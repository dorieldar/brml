\documentclass[20pt]{extarticle}
\usepackage{amsmath}
\DeclareMathOperator{\E}{\mathbb{E}}
\DeclareMathOperator{\HH}{\mathbb{H}}

\usepackage{graphicx}
\usepackage{hyperref}
\usepackage[latin1]{inputenc}

\begin{document}
ex 8.5\\ 
1.\\
oddness:\\
\begin{equation}
\int_{-\infty}^{\infty}e^{-\frac{1}{2}r^2}rdr = 0
\end{equation}
by change of variable:\\
\begin{equation}
\int_{-\infty}^{\infty}e^{-\frac{1}{2}\frac{(r-\mu)^2}{\sigma^2}}\frac{(r-\mu)}{\sigma}dr = 0
\end{equation} 
compute the easy side of the equation (and ignore $\sigma$):\\
\begin{equation}\label{easy}
\mu\int_{-\infty}^{\infty}e^{-\frac{1}{2}\frac{(r-\mu)^2}{\sigma^2}}dr = \mu\sqrt{2\pi\sigma^2}
\end{equation}
and so from \ref{easy}:\\
\begin{equation}
\int_{-\infty}^{\infty}e^{-\frac{1}{2}\frac{(r-\mu)^2}{\sigma^2}}rdr = \mu\sqrt{2\pi\sigma^2}
\end{equation} 


Same line of arguments shows the results for the multivariate case
$$\E(N(,\mu,\Sigma)) = \mu$$
2. \\
by change of variable it is enough to compute:\\
$$\int_{-\infty}^{\infty}e^{-\frac{1}{2}\frac{r^2}{\sigma^2}}r^2dr $$

using integration by parts: \\
$$dv = e^{-\frac{1}{2}\frac{r^2}{\sigma^2}}r , u = r$$
$$ v = -\sigma^2 e^{-\frac{1}{2}\frac{r^2}{\sigma^2}}$$
$uv$ is odd so:\\

$$\int dvu= 0 + \sigma^2 \int_{-\infty}^{\infty}e^{-\frac{1}{2}\frac{r^2}{\sigma^2}} = \sigma^2 \sqrt{2\pi\sigma^2}$$
Same line of arguments shows this for multivariate (as cov is symetric diff and mult operation are identical to the scalae case)

ex 8.6\\
by change of variable move $\mu$ to linear component and compute the equivalent\\
(assumption A is symmetric ??)\\
$$\E(x-\mu)^TA(x-\mu)|_{N(0,\Sigma)} = \E x^{T}Ax + \mu^TA\mu + 0$$
$$\E x^{T}Ax = \E trace(x^{T}xA) = trace(\E x^{T}xA) $$
last equation since expectation is a scalar\\
$$trace(\E x^{T}xA) = trace(\Sigma A)$$

8.11\\
find $A$ such that  $A^TA = \Sigma$ due to symmetry\\
change of variable: $$z = A(x-\mu)$$ $$dz = det(A)dx = \sqrt{det(\Sigma)}dx$$
use resulat for 1 dimensional
$$\sqrt{det(\Sigma)}\int \exp^{z^Tz}dz = \sqrt{(2\pi)^n}\sqrt{det(\Sigma)}=\sqrt{det(2\pi\Sigma)}}$$

8.13\\
for both metrics assume $\mu=0$ the $skewness=0$ due to oddness of integrand, for curtosis use integration by parts to derive result. \\

8.20\\
by change of variable assume $\mu = 0$ \\
$$D = \int e^{-\frac{1}{2}x^{T}\Sigma^{-1}x}dx = \sqrt{det(2\pi\Sigma)}$$
$$H = -D^{-1} \int (-\frac{1}{2}x^{T}\Sigma^{-1}x- \log{D})e^{-\frac{1}{2}x^{T}\Sigma^{-1}x}dx$$
$$H = \frac{D^{-1}}{2}\int x^{T}\Sigma^{-1}xe^{-\frac{1}{2}x^{T}\Sigma^{-1}x}dx + D^{-1}\log{D}\int e^{-\frac{1}{2}x^{T}\Sigma^{-1}x}dx$$
$$H = \frac{D^{-1}}{2}\int x^{T}\Sigma^{-1}xe^{-\frac{1}{2}x^{T}\Sigma^{-1}x}dx + \frac{\log(2\pi\Sigma)}{2}$$
using result 8.5 for expectation of quadartic form $x^T\Sigma^{-1}x$: 
$$H = \frac{1}{2}trace(I) + \frac{\log(2\pi\Sigma)}{2}$$
$$H = \frac{Dim(x)}{2} + \frac{\log(2\pi\Sigma)}{2}$$
$$H = \frac{Dim(x)}{2}  + \frac{\log(2\pi\Sigma)}{2}$$

8.22\\
1.\\
$$KL(\delta) = \sum{\log{P(x|\theta)} - \log{P(x|\theta + \delta)})P(x|\theta)}$$
Taylor :
$$KL(0) = 0 $$
$$KL'(\delta)=-\sum{\frac{P(x|\theta)}{P(x|\theta + \delta)}\frac{\partial P}{\partial\delta}}$$
$$KL'(0) = -\sum\frac{P(x|\theta + \delta) - P(x|\theta)}{\delta}= 
\frac{1}{\delta}(\sum P(x|\theta + \delta) - \sum P(x|\theta) = 0$$

hence the taylor expansion is from order 2 \\
8.30 \\
$$p(\theta |X) \approx p(X|\theta)$$
$$\log p(X|\theta)= \sum_i \log \frac{e^{-\lambda}\lambda^{x_i}}{x_i!} = \sum_i (-\lambda + x_i\log \lambda - \log x_i!) =$$
$$-n\lambda + \log \lambda \sum_i x_i + const(\lambda)$$
$$F'(\lambda)= -n + \frac{\sum_i x_i}{\lambda}$$
$$argmax(F(\lambda))= \frac{\sum_i x_i}{n}$$

8.31\\
mean: linearity of expectation
variance:
$$\E x^{T}x = \sum_{i}p_{i} \E x^{T}x |_{N(\Sigma_i,\mu_i)} =$$
$$\sum_{i}p_i\E(x-\mu_i)^T(x-\mu_i)|_{N(\Sigma_i,0)}=$$
$$\sum_{i}p_i(\Sigma_i + \mu_i^{T}\mu_i)$$
8.32 \\
$$ZZ^T = NS^{-1}U^TYY^TUS^{-1}$$
$$YY^T = USV^TVSU^T = US^2U^T$$
$$ZZ^T= NI$$
8.34\\
simple
\end{document}
